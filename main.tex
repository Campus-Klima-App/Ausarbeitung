\renewcommand{\familydefault}{\sfdefault}

\documentclass{article}
\usepackage{geometry} % Für Layout
\usepackage{geometry}
 \geometry{
 a4paper,
 left=35mm,
 right=50mm,
 top=25mm,
 bottom=25mm,
 footnotesep=20mm,
 }
\usepackage[utf8]{inputenc}
\usepackage{float}
\usepackage[
backend=biber,
style=alphabetic,
]{biblatex}

\addbibresource{references.bib} %Imports bibliography file

\usepackage[T1]{fontenc}
\usepackage{hyperref}
\usepackage[ngerman]{babel}
\usepackage{csquotes} % Für Anführungszeichen
\usepackage{hyphenat}
\usepackage{caption}
\usepackage{lipsum} % für Linespacing
\usepackage{setspace} % für Linespacing
\usepackage{footmisc} % für multiple referenced Footnotes
\usepackage{graphicx}
\usepackage{tabularx} % für schöner umgebrochene Tabellen

\hyphenation{Mathe-matik wieder-gewinnen} % Für korrekte Silbentrennung

\title{LoRaWAN Projekt: Sensortree}
\author{Jakob Weirich (\textit{Matrikelnummer}), Michel Schwarz (\textit{Matrikelnummer}) \\und Robert Deppe (\textit{649819})}


\onehalfspacing 

\begin{document}
\pagenumbering{gobble}  % Keine Seitenzahlen
\onehalfspacing
\maketitle
\newpage

\tableofcontents
\newpage

\subsection*{Aufgabenstellung}
Aufgabenstellung

\vfill{}

\begin{abstract}
    Abstract
\end{abstract}

\vfill{}

\newpage

\setcounter{page}{1}
\pagenumbering{arabic}

\section{Einleitung}
Einleitung
\newpage

\section{Beschreibung der Ergebnisse}


\subsection{Überblick}

\section{Fazit}

\newpage
\section{Ausblick}

\newpage

\listoffigures
\newpage

\printbibliography
\newpage

\pagenumbering{gobble}  % Keine Seitenzahlen
\section*{Eidesstattliche Erklärung}
Hiermit erklären wir, Jakob Weirich, Michel Schwarz und Robert Deppe dass wir die vorliegende Arbeit selbstständig und ohne unerlaubte Hilfe angefertigt, andere als die angegebenen Quellen und Hilfsmittel nicht benutzt und die den benutzten Quellen wörtlich oder inhaltlich entnommenen Stellen als solche kenntlich gemacht haben.


\newpage

\end{document}
